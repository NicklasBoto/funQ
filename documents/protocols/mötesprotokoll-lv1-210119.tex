\documentclass[11pt,a4paper]{article}
\usepackage[utf8]{inputenc}
\usepackage[swedish]{babel}

% Skriv in namn efter roller.
% Föredragen ordning av namn: enligt bokstavsordning av efternamn.
\newcommand{\secretary}{Beata Burreau}
\newcommand{\director}{Fabian Forslund}
\newcommand{\partaking}{Beata Burreau, Matilda Blomqvist, Nicklas Botö, 
                        Fabian Forslund, Marcus Jörgensson, Joel Rudsberg}

\usepackage[parfill]{parskip}

\title{Mötesprotokoll}
\begin{document}

\maketitle
\noindent
\textbf{Mötesordförande:} \director\\
\noindent
\textbf{Sekreterare:} \secretary\\
\noindent
\textbf{Deltagande:} \partaking
\section{Mötets gång}

Mötet inleddes med en kort avstämning av gruppmedlemmarnas bakgrund och tidigare erfarenhet 
inom/intressen för kvantdatorer och funktionell programmering. Fabian och Nicklas berättade lite om 
bakgrunden till projektet.

På mötet satte vi upp ett gemensamt Overleafprojekt, en gemensam kalender i Google Calendar och 
samtliga gruppmedlemmar har fått tillgång till en gemensam Drivemapp, Github och Trello board. 

Vi valde onsdagar kl 13:15 - 15:00 som tid för ett stående, lite längre gruppmöte. Utöver det är 
gruppen överens om att ha fler korta avstämningsmöten per vecka, tid för dessa är ej satt än. 

Fabian valdes till projektledare. Sekreterarrollen kommer rotera veckovis mellan samtliga 
gruppmedlemmar, med start denna vecka (LV1) då Beata är sekreterare. 

Vi valde att rapporten skulle skrivas på engelska, samt att skapa en mapp i git-repot för rapporten 
och dylika dokument.

\section{Inför nästa möte}

Till nästa möte (20/1) ska alla ha bekantat sig med de grundläggande begreppen inom fältet, t.ex. 
genom att läsa A lambda calculus for quantum computation with classical control av Selinger och Valiron. 

Vi ska även kontakta Robin Adams för att sätta upp ett första handledningsmöte. 


\end{document}
